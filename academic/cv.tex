%______________________________________________________________________________________________________________________
% @brief    based on LaTeX2e Resume by Kamil K Wojcicki
\documentclass[margin,line]{resume}

\usepackage{hyperref}
%______________________________________________________________________________________________________________________
\begin{document}
\name{\Large Michael Kocher}
\begin{resume}

    %__________________________________________________________________________________________________________________
    % Contact Information
    \section{\mysidestyle Contact\\Information}

    San Francisco, CA\hfill e-mail: michael.kocher@me.com  \vspace{0mm}\\\vspace{-4.5mm}%

    % end Contact
    % end -----------------------------------------------

    
    % Education
    \section{\mysidestyle Education}

    \textbf{Ph.D. Material Science and Engineering} \\
    Arizona State University,  Tempe, AZ 85287 \hfill December 2008\\
    Thesis: \textsl{The Electronic Structure of Lithium Transition Metal Oxides}\vspace{2mm}\\
    \textbf{Bachelor of Science in Physics} \\
    Arizona State University,  Tempe, AZ 85287 \hfill April 2004\vspace{2mm}\\
    \textbf{Bachelor of Science in Chemistry}\\
    Arizona State University,  Tempe, AZ 85287 \hfill April 2004\\\vspace{-3mm}

    % end Education
    % end -----------------------------------------------



    % Tools and Frameworks
    \section{\mysidestyle Programming, Tools and Frameworks}

    \textbf{Languages}: Python, Scala, Ruby\vspace{2mm}\\
                Extensive experience with PBS, Slurm and SGE for running high-throughput
                computations on HPC systems. MongoDB and SQL for database driven
                applications and experienced with object-oriented programming using
                Python and Ruby. Skilled in several web frameworks, such as Django, Flask, Sinatra for designing web
                applications. 8 years of experience with Vienna
                \textsl{ab initio} Simulation Package (VASP) and Wien2k to compute properties of materials from first principles. Proficient with Bash for general scripting and system administration. Git for source code version control.


  % End tools Frameworks
  % end -----------------------------------------------


   % Research Interests
    \section{\mysidestyle Research\\Interests}

    High-throughput computing, workflows, \textsl{ab initio} materials science
    and chemistry, document based data storage (MongoDB),
    machine learning, message queues, scientific gateways and RESTful interfaces for web driven scientific applications.

  % Research Interests 
  % end -----------------------------------------------

  % Awards ----------------------------------------------------------------
    \section{\mysidestyle Awards}
    LBNL Spot Prize: ``For outstanding and inspiring
      work on the Materials Project'' \hfill January 2012\vspace{2mm}\\
    Top Prize in the LBNL 2010 Mobility Contest: \hfill February 2010\\
      ``Mobility Materials Genome: iPhone app to
       accelerate materials design process''

  % end Awards
  % end -----------------------------------------------
    
  % Professional Experience
       \section{\mysidestyle Professional\\Experience}
    \textbf{Postdoctoral Chemist Researcher} \hfill January 2009 -- April 2012\vspace{1mm}\\
		Environmental Energy Technology Division \\
		Lawrence Berkeley National Lab, Berkeley, CA\\
		Advisor: Dr. Kristin Persson\\\vspace{-1.5mm}
    \begin{list2}
			\item Materials Project (\url{http://www.materialsproject.org}) lead developer and core database administrator
			\item Designed a distributed high-throughput workflow engine to perform MPI VASP calculations across several HPC centers
      \item designed NoSQL database driven scientific applications using Django and MongoDB
		  \item Performed calculation to investigate the role of Al substitution in Li transition metal oxides to understand Li diffusion, cation ordering and stability
		 \item Phonon calculations using \textsl{ab initio} quantum molecular dynamic simulations on $\mathrm{LiMnO_2}$ and $\mathrm{LiMn_2O_4}$ to determine thermal free energies and diffusion properties from first principles
     \item Performed electronic structure calculations on $\mathrm{LiC_x}$ and various Li transition metal oxides to understand the charge compensation mechanism during delithation
		\end{list2}

    % Graduate research
		\textbf{Graduate Research Assistant} \hfill September 2004 -- December 2008\vspace{1mm}\\
		Center for Solid State Science\\
Arizona State University Tempe\\
Advisor: Prof. Peter Rez\\\vspace{-1.5mm}
		\begin{list2}
			\item Electronic Structure of $\mathrm{LiMn_{1/2}Ni_{1/2}O_2}$ and $\mathrm{LiMn_{1/3}Ni_{1/3}Co_{1/3}O_2}$
			\item Electronic Structure calculations on $\mathrm{LiFePO_4}$ and $\mathrm{FePO_4}$
			\item Surface interactions of Aspartic and Glutamic Acid on $\mathrm{Ca(C_2O_4)(H_2O)}$
			\item Calculations on 5,7 defects in single wall carbon nanotubes
		\end{list2}


    \newpage

    % undergraduate research

		\textbf{Undergraduate Research Assistant} \hfill August 2001 -- August 2004\vspace{1mm}\\
		Center for Solid State Science\\
Arizona State University Tempe, AZ\\
Advisor: Prof. Andrew Chizmeshya\\\vspace{-1.5mm}
		\begin{list2}
			\item Performed calculations of $\mathrm{MgCO_3}$ and $\mathrm{Mg_2SiO_4}$ to determine elastic constants and free energies to understand chemical reactions in mineral sequestration of $\mathrm{CO_2}$
		 \item Spin polarized calculation to determine the enthalpy of mixing for $\mathrm{Mg_{1-x}Fe_{x}CO_{3}}$ solid solution to understand the effect of Fe impurities in serpentine on the sequestration process.
		 \item Phonon calculations using quantum molecular dynamic simulations on $\mathrm{MgCO_3}$ and $\mathrm{Mg_2SiO_4}$ to determine thermal free energies from first principles
		\end{list2}
		
    % Undergrad Research
		\textbf{Undergraduate Research Assistant} \hfill May 2002 – August 2004\vspace{1mm}\\
		Center for Solid State Science\\
Arizona State University Tempe, AZ\\
Advisor: Prof. Peter Rez\\\vspace{-1.5mm}
		\begin{list2}
			\item Electron Energy Loss Spectroscopy calculations of $\mathrm{SrTiO_3}$ and $\mathrm{LaTiO_3}$ from \textsl{ab initio} methods
			\item Calculations of surface energies of $\mathrm{Ca(C_2O_4)(H_2O)}$ and $\mathrm{Ca(C_2O_4)(H_2O)_2}$
			\item Role of acetylene on Ni (110) and (110) and Ni on graphene sheets for understanding of carbon nanotube formation
		\end{list2}
		
    % Undergrad Research
		\textbf{Undergraduate Research Assistant} \hfill August 2000 - May 2001\vspace{1mm}\\
		Department of Chemistry\\
Arizona State University Tempe, AZ\\
Advisor: Prof. Tim Steimle\\\vspace{-1.5mm}
		\begin{list2}
		\renewcommand{\labelitemi}{$\star$}
		\item Assisted in High resolution Spectroscopy experiments
		\item Maintenance of laboratory apparatus
		\item Programming of EEPROM Pic devices
		\end{list2}

  % Professional Experience
  % end -----------------------------------------------

  % Selected Publications ------------------------------------------------
  \section{\mysidestyle Selected Publications}
    M. Kocher and K. Persson, "Li Mobility and the Electronic Structure of $\mathrm{LiMn_{1/3}Ni_{1/3}Co_{1-x}Al_{x}O_{2}}$ Determined from First-Principles" \textsl{Physical Review B} (submitted)\vspace{2mm}\\
    S. Miao, M. Kocher, P. Rez, B. Fultz, R. Yazami, and C. C. Ahn, "Local electronic structure of olivine phases of $\mathrm{Li_xFePO_4}$", \textsl{The Journal of Physical Chemistry A}, vol.111, pp.4242–7, May (2007).\vspace{2mm}\\
    S . Miao, M. Kocher, P. Rez, B. Fultz, Y. Ozawa, R. Yazami, and C. Ahn, "Local electronic structure of layered $\mathrm{Li_xNi_{0.5}Mn_{0.5}O_2}$ and $\mathrm{Li_{x}Ni_{1/3}Mn_{1/3}Co_{1/3}O_2}$", \textsl{Journal of Physical Chemistry B}, vol. 109, no. 49, pp. 23473–23479 (2005).\vspace{2mm}\\
    M.P. Kocher, D.A. Muller and P. Rez,, "The Oxygen K Edge in Strontium Titanate and Lanthanum Titanate", \textsl{Microscopy and Microanalysis}, (suppl. 2), 9, 842 (2003).\vspace{2mm}\\
    A. V. G. Chizmeshya, M. J. McKelvy, G. Wolf, M. Kocher, D. Gormley, "Quantum Simulations Studies of Olivine Mineral Carbonation", \textsl{Proc. 28th International Technical Conference on Coal Utilization   \& Fuel Systems} 319-330 (2003).


		\section{\mysidestyle Conference Presentations and
                  Posters}
                ``Using Python to Accelerate Materials Design'',
                Michael Kocher, Dan Gunter and Shreyas Cholia, PyCon,
                Santa Clara, CA March (2012)\vspace{2mm}\\
		``Using MongoDB for Materials Discovery'' Michael
                Kocher and Dan Gunter, MongoSV, Santa Clara, CA December (2011)\vspace{2mm}\\
                ``The Effect of Al Substitution in Lithium Transition Metal Oxides from First-principles'' Michael Kocher, Kristin A. Persson and Quentin M. Ramasse. Materials Research Society Spring Meeting, San Francisco, CA April (2010)\vspace{2mm}\\
		``Does Ni change its charge state in $\mathrm{Li(Mn_{0.5}Ni_{0.5})O_2}$ cathode materials?'' Peter Rez and Michael Kocher. 208 Electrochemical Society Meeting, Los Angeles, California, October (2005) \vspace{2mm}\\
		``Local Electronic Structure of Layered $\mathrm{Li_{x}Mn_{1/3}Ni_{1/3}Co_{1/3}O_{2}}$'' Brent Fultz, Shu Miao, Michael Kocher, Peter Rez, Yasuroni Ozawa, Rachid Yazami, and Channing Ahn. 208 Electrochemical Society Meeting, Los Angeles, California, October (2005) \vspace{2mm}\\

    \newpage

		``$\mathrm{CO_2}$ Mineral Carbonation Processes in
                  Olivine Feedstock: Insights from the Atomic Scale
                  Simulation'' Andrew V.G. Chizmeshya, Michael J.
                McKelvy, Deirdre Gormley, Michael Kocher, Ryan Nunez,
                Young-Chul Kim and Ray Carpenter, Proc. 29th
                International Technical Conference on Coal Utilization
                and Fuel Systems, Clearwater, Florida, April (2004)
                \vspace{2mm}\\
		``Exploration of the Role of Heat Activation in Enhancing Serpentine Carbon Sequestration Reactions'' Michael J. McKelvy, Andrew V.G. Chizmeshya, Jason Diefenbacher, George Wolf, Brandon Doss, Deirdre Gormley, Michael Kocher, and Hamdallah Bearat, Proc. 29th International Technical Conference on Coal Utilization and Fuel Systems, Clearwater, Florida, April (2004)\vspace{2mm}\\
    %__________________________________________________________________________________________________________________
                \section{\mysidestyle Workshops}
		Summer School of Computational Material Science \hfill July 2006\\
		Ab Initio Molecular Dynamics Simulation Methods in Chemistry\\
		University of Illinois Urbana-Champaign\vspace{-2mm}\\\vspace{-2mm}

		International Center for Materials Research Summer School\hfill August 2005\\
		First Principles Calculations for Condensed Matter and Nanoscience\\
		University of California Santa Barbara\\

  % Pubs
  % end -----------------------------------------------


    %__________________________________________________________________________________________________________________
    % Referees
		\section{\mysidestyle References}
		{\sl Available on request.}
%______________________________________________________________________________________________________________________
%______________________________________________________________________________________________________________________
\end{resume}
\end{document}


%______________________________________________________________________________________________________________________
% EOF
